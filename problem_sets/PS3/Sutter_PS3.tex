\documentclass[12pt,letterpaper]{article}
\usepackage{graphicx,textcomp}
\usepackage{natbib}
\usepackage{setspace}
\usepackage{fullpage}
\usepackage{color}
\usepackage[reqno]{amsmath}
\usepackage{amsthm}
\usepackage{fancyvrb}
\usepackage{amssymb,enumerate}
\usepackage[all]{xy}
\usepackage{endnotes}
\usepackage{lscape}
\newtheorem{com}{Comment}
\usepackage{float}
\usepackage{hyperref}
\newtheorem{lem} {Lemma}
\newtheorem{prop}{Proposition}
\newtheorem{thm}{Theorem}
\newtheorem{defn}{Definition}
\newtheorem{cor}{Corollary}
\newtheorem{obs}{Observation}
\usepackage[compact]{titlesec}
\usepackage{dcolumn}
\usepackage{tikz}
\usetikzlibrary{arrows}
\usepackage{multirow}
\usepackage{subcaption}
\usepackage{xcolor}
\newcolumntype{.}{D{.}{.}{-1}}
\newcolumntype{d}[1]{D{.}{.}{#1}}
\definecolor{light-gray}{gray}{0.65}
\usepackage{url}
\usepackage{listings}
\usepackage{color}

\definecolor{codegreen}{rgb}{0,0.6,0}
\definecolor{codegray}{rgb}{0.5,0.5,0.5}
\definecolor{codepurple}{rgb}{0.58,0,0.82}
\definecolor{backcolour}{rgb}{0.95,0.95,0.92}

\lstdefinestyle{mystyle}{
	backgroundcolor=\color{backcolour},   
	commentstyle=\color{codegreen},
	keywordstyle=\color{magenta},
	numberstyle=\tiny\color{codegray},
	stringstyle=\color{codepurple},
	basicstyle=\footnotesize,
	breakatwhitespace=false,         
	breaklines=true,                 
	captionpos=b,                    
	keepspaces=true,                 
	numbers=left,                    
	numbersep=5pt,                  
	showspaces=false,                
	showstringspaces=false,
	showtabs=false,                  
	tabsize=2
}
\lstset{style=mystyle}
\newcommand{\Sref}[1]{Section~\ref{#1}}
\newtheorem{hyp}{Hypothesis}

\title{Chloe Sutter / Problem Set 3}
\date{Due: February 17, 2020}
\author{QTM 200: Applied Regression Analysis}

\begin{document}

\maketitle	
\noindent In this problem set, you will run several regressions and create an add variable plot (see the lecture slides) in \texttt{R} using the \texttt{incumbents\_subset.csv} dataset. Include all of your code.

	\vspace{.5cm}
	\section*{Question 1 (20 points)}
	\vspace{.25cm}
	\noindent We are interested in knowing how the difference in campaign spending between incumbent and challenger affects the incumbent's vote share. 
	\begin{enumerate}
		\item Run a regression where the outcome variable is \texttt{voteshare} and the explanatory variable is \texttt{difflog}.
		\lstinputlisting[language=R, firstline=50, lastline=60]{PS3_Sutter.R}  
		\vspace{.25cm}
		
		\item Make a scatterplot of the two variables and add the regression line. 
		\lstinputlisting[language=R, firstline=63, lastline=66]{PS3_Sutter.R}  
		\vspace{.25cm}
		
		\newpage
		
		\begin{figure}[h!]\centering
		\caption{\footnotesize }
		\label{fig:plot_1}
		\includegraphics[width=.85\textwidth]{Regress1.pdf}
		\end{figure}
		\vspace{.25cm}
		
		\item Save the residuals of the model in a separate object.	
		\lstinputlisting[language=R, firstline=72, lastline=73]{PS3_Sutter.R}  
		\vspace{.25cm}
		
		\item Write the prediction equation.
		\vspace{.15cm}
		
		Y = $\alpha$ + $\beta$ X
		
		\vspace{.1cm}
		
		Y = .57 + .04X where Y is a vector of observed outcomes for voteshare and and X is a vector of difflog. 
		
	\end{enumerate}
	

	
	\section*{Question 2 (20 points)}
\noindent We are interested in knowing how the difference between incumbent and challenger's spending and the vote share of the presidential candidate of the incumbent's party are related.	\vspace{.25cm}
	\begin{enumerate}
		\item Run a regression where the outcome variable is \texttt{presvote} and the explanatory variable is \texttt{difflog}.
		\lstinputlisting[language=R, firstline=89, lastline=94]{PS3_Sutter.R}  
		\vspace{.25cm}

		\item Make a scatterplot of the two variables and add the regression line. 	
		\lstinputlisting[language=R, firstline=97, lastline=99]{PS3_Sutter.R}  
		
		\begin{figure}[h!]\centering
		\caption{\footnotesize }
		\label{fig:plot_1}
		\includegraphics[width=.85\textwidth]{regress2.pdf}
		\end{figure}
		\vspace{.25cm}
		
		\newpage
		
		\item Save the residuals of the model in a separate object.
		\lstinputlisting[language=R, firstline=105, lastline=106]{PS3_Sutter.R}  
		
		\item Write the prediction equation.
		
		Y = $\alpha$ + $\beta$X
		
		\vspace{.15cm}
	
		Y = .507 + .023X where Y is a vector of the observed outcomes for presvote and X is a vector of difflog.
  
	\end{enumerate}

	
\section*{Question 3 (20 points)}

\noindent We are interested in knowing how the vote share of the presidential candidate of the incumbent's party is associated with the incumbent's electoral success.
	\vspace{.25cm}
	\begin{enumerate}
		\item Run a regression where the outcome variable is \texttt{voteshare} and the explanatory variable is \texttt{presvote}.
		\lstinputlisting[language=R, firstline=122, lastline=126]{PS3_Sutter.R} 
		\vspace{.25cm} 
			
		\item Make a scatterplot of the two variables and add the regression line. 
		\lstinputlisting[language=R, firstline=129, lastline=131]{PS3_Sutter.R}  
		
		\begin{figure}[h!]\centering
		\caption{\footnotesize }
		\label{fig:plot_1}
		\includegraphics[width=.85\textwidth]{regress3.pdf}
		\end{figure}
		\vspace{.25cm}
		
		\newpage

		\item Write the prediction equation.
		
		Y = $\alpha$ + $\beta$X 
		
		\vspace{.1cm}
		
		Y = .44 + .38X where Y is a vector of observed outcomes for voteshare and X is a vector of presvote.
		
	\end{enumerate}
	

\section*{Question 4 (20 points)}
\noindent The residuals from part (a) tell us how much of the variation in \texttt{voteshare} is $not$ explained by the difference in spending between incumbent and challenger. The residuals in part (b) tell us how much of the variation in \texttt{presvote} is $not$ explained by the difference in spending between incumbent and challenger in the district.
	\begin{enumerate}
		\item Run a regression where the outcome variable is the residuals from Question 1 and the explanatory variable is the residuals from Question 2.
		\lstinputlisting[language=R, firstline=161, lastline=171]{PS3_Sutter.R} 
		\vspace{.25cm} 
			
		\item Make a scatterplot of the two residuals and add the regression line. 
		\lstinputlisting[language=R, firstline=174, lastline=176]{PS3_Sutter.R} 
		
		\begin{figure}[h!]\centering
		\caption{\footnotesize }
		\label{fig:plot_1}
		\includegraphics[width=.85\textwidth]{regress4.pdf}
		\end{figure}
		\vspace{.25cm}
		
		\newpage
		
		\item Write the prediction equation.
		
		Y = $\alpha$ + $\beta$X 
		
		\vspace{.1cm}
		
		Y = -0.0000000000000000048 + 0.25X
		Where Y = residuals describing how much of the variation in VOTESHARE is not explained by the difference in spending between incumbent and challenger
		and where X = residuals describing how much of the variation in PRESVOTE is not explained by the difference in spending between incumbent and challenger in the district.
	\end{enumerate}
	
	\vspace{1cm}

\section*{Question 5 (20 points)}
\noindent What if the incumbent's vote share is affected by both the president's popularity and the difference in spending between incumbent and challenger? 
	\begin{enumerate}
		\item Run a regression where the outcome variable is the incumbent's \texttt{voteshare} and the explanatory variables are \texttt{difflog} and \texttt{presvote}.
		Find regression by-hand.
		\lstinputlisting[language=R, firstline=193, lastline=221]{PS3_Sutter.R} 
		
		Check regression in r
		\lstinputlisting[language=R, firstline=223, lastline=230]{PS3_Sutter.R} 
		
		\item Write the prediction equation.
		
		\vspace{.1cm}
		
		$\mu$y = $\beta$0 + $\beta$1X1 + $\beta$2X2
		
		\vspace{.1cm}
		
		$\mu$y = 0.449 + 0.035X1 + 0.256X2 where X1 represents difflog and X2 represents presvote.
		
		\item What is it in this output that is identical to the output in Question 4? Why do you think this is the case?
		Reflect on your finding. Don't write anything. Just think about it.
	\end{enumerate}




\end{document}
